\chapter{Activities of the Department}

Overview of the Functional Unit: In this chapter, students should describe the specific department or team they were embedded in during the internship. While Chapter 1 focused on the entire company, this section focuses on the immediate work environment.

\section{Departmental Role and Objectives}
Define the primary purpose of your department (e.g., Embedded Systems Team, VLSI Design Group, or Quality Assurance). Explain how this department contributes to the company’s overall product lifecycle.

\section{Technical Infrastructure} 
Describe the tools, hardware, and software environment used within the department. Mention specific lab equipment, servers, or proprietary software frameworks utilized by the team.

\section{Workflow and Methodologies} 
Detail the operational processes followed, such as the Software Development Life Cycle (SDLC), Agile Sprints, or hardware testing protocols.

%%%%%%%%%%%%%%%%%%%%%%%%%%%%%%%%%%%%%%%%%%%%%%%%%%%%%%%%%%%%%%%%%%%%%%%%%%%%%%%%%%%%%%%%%%%%%%%%%%%%
% NOTE: This section is meant to teach how to use "Acronyms" and "Glossaries" in the report. 
% 		So Remove or comment the following section and it's content entirely.
%%%%%%%%%%%%%%%%%%%%%%%%%%%%%%%%%%%%%%%%%%%%%%%%%%%%%%%%%%%%%%%%%%%%%%%%%%%%%%%%%%%%%%%%%%%%%%%%%%%%
\section{Use of Acronyms and Glossaries}
Acronyms are nothing but the short form of regular repeated word. Say for example, you have a repeat word "Integrated Circuits" and you want to use a short form for it as "IC". For which you have to first define the word and use it wherever you wanted to refer it.

First, let's look at the definition, which has to be entered in \texttt{Glossaries.tex} under \texttt{CoverPages} directory.
\begin{verbatim}
	%\newacronym{<Ref>}{<Short-Form>}{<Expanded word>}
	\newacronym{ic}{IC}{Integrated Circuits}
\end{verbatim}
In order to use the defined acronym, use the commands \verb|\gls{<Ref>}| as shown below

%As an example, call the definition with \verb|\gls{ic}| and the outcome of it is reflected as, \gls{ic}.

Note: For the First time, the expanded form appears along with the Short-form definition inside parenthesis. But when the \verb|\gls{}| is repeated, only Short-form appears inside the parenthesis.

Now, let's look at the definition of symbols. Follow the syntax to define the symbol first, inside \texttt{Glossaries.tex} under \texttt{CoverPages} directory.
\begin{verbatim}
	%\newglossaryentry{<Ref>}{name=<Symbol>, description={<description about the symbol>}, type=<List type>}
	\newglossaryentry{rc}{name=$\tau$, description={Time constant}, type=symbolList}
\end{verbatim}

%As an example, the rate of change is defined with \verb|\gls{rc}| and the outcome of it is reflected as, the rate of change is defined with \gls{rc}.